%!TEX TS-program = xelatex
%!TEX encoding = UTF-8 Unicode
% Awesome CV LaTeX Template for CV/Resume
%
% This template has been downloaded from:
% https://github.com/posquit0/Awesome-CV
%
% Author:
% Claud D. Park <posquit0.bj@gmail.com>
% http://www.posquit0.com
%
%
% Adapted to be an Rmarkdown template by Mitchell O'Hara-Wild
% 23 November 2018
%
% Template license:
% CC BY-SA 4.0 (https://creativecommons.org/licenses/by-sa/4.0/)
%
%-------------------------------------------------------------------------------
% CONFIGURATIONS
%-------------------------------------------------------------------------------
% A4 paper size by default, use 'letterpaper' for US letter
\documentclass[11pt,a4paper,]{awesome-cv}

% Configure page margins with geometry
\usepackage{geometry}
\geometry{left=1.4cm, top=.8cm, right=1.4cm, bottom=1.8cm, footskip=.5cm}


% Specify the location of the included fonts
\fontdir[fonts/]

% Color for highlights
% Awesome Colors: awesome-emerald, awesome-skyblue, awesome-red, awesome-pink, awesome-orange
%                 awesome-nephritis, awesome-concrete, awesome-darknight

\definecolor{awesome}{HTML}{1422BE}

% Colors for text
% Uncomment if you would like to specify your own color
% \definecolor{darktext}{HTML}{414141}
% \definecolor{text}{HTML}{333333}
% \definecolor{graytext}{HTML}{5D5D5D}
% \definecolor{lighttext}{HTML}{999999}

% Set false if you don't want to highlight section with awesome color
\setbool{acvSectionColorHighlight}{true}

% If you would like to change the social information separator from a pipe (|) to something else
\renewcommand{\acvHeaderSocialSep}{\quad\textbar\quad}

\def\endfirstpage{\newpage}

%-------------------------------------------------------------------------------
%	PERSONAL INFORMATION
%	Comment any of the lines below if they are not required
%-------------------------------------------------------------------------------
% Available options: circle|rectangle,edge/noedge,left/right

\photo{./image/me.jpg}
\name{Pete}{Apicella}

\position{Molecular Biologist}
\address{13650 E Colfax Avenue, Aurora, Colorado}

\email{\href{mailto:apicellapv@gmail.com}{\nolinkurl{apicellapv@gmail.com}}}
\orcid{0000-0002-0747-5338}
\github{apicellap}
\linkedin{peterapicella}

% \gitlab{gitlab-id}
% \stackoverflow{SO-id}{SO-name}
% \skype{skype-id}
% \reddit{reddit-id}

\quote{I am a highly published molecular biologist with experience in
plant and fungal biology. I bring startup environment experience}

\usepackage{booktabs}

\providecommand{\tightlist}{%
	\setlength{\itemsep}{0pt}\setlength{\parskip}{0pt}}

%------------------------------------------------------------------------------



% Pandoc CSL macros
\newlength{\cslhangindent}
\setlength{\cslhangindent}{1.5em}
\newlength{\csllabelwidth}
\setlength{\csllabelwidth}{3em}
\newenvironment{CSLReferences}[3] % #1 hanging-ident, #2 entry spacing
 {% don't indent paragraphs
  \setlength{\parindent}{0pt}
  % turn on hanging indent if param 1 is 1
  \ifodd #1 \everypar{\setlength{\hangindent}{\cslhangindent}}\ignorespaces\fi
  % set entry spacing
  \ifnum #2 > 0
  \setlength{\parskip}{#2\baselineskip}
  \fi
 }%
 {}
\usepackage{calc}
\newcommand{\CSLBlock}[1]{#1\hfill\break}
\newcommand{\CSLLeftMargin}[1]{\parbox[t]{\csllabelwidth}{#1}}
\newcommand{\CSLRightInline}[1]{\parbox[t]{\linewidth - \csllabelwidth}{#1}}
\newcommand{\CSLIndent}[1]{\hspace{\cslhangindent}#1}

\begin{document}

% Print the header with above personal informations
% Give optional argument to change alignment(C: center, L: left, R: right)
\makecvheader

% Print the footer with 3 arguments(<left>, <center>, <right>)
% Leave any of these blank if they are not needed
% 2019-02-14 Chris Umphlett - add flexibility to the document name in footer, rather than have it be static Curriculum Vitae
\makecvfooter
  {July 2022}
    {Pete Apicella~~~·~~~CV}
  {\thepage}


%-------------------------------------------------------------------------------
%	CV/RESUME CONTENT
%	Each section is imported separately, open each file in turn to modify content
%------------------------------------------------------------------------------



\hypertarget{education}{%
\section{Education}\label{education}}

\begin{cventries}
    \cventry{MSc Plant Sciences}{University of Connecticut}{Storrs, Connecticut}{2020}{\begin{cvitems}
\item Thesis - Studies of the Cannabinoid Biosynthetic Pathway in Developing Cannabis sativa Flowers and Elucidation of Genetic and Physiological Mechanisms Regulating Cannabinoid Production
\end{cvitems}}
    \cventry{BSc Horticulture}{University of Connecticut}{Storrs, Connecticut}{2018}{\begin{cvitems}
\item Honors Thesis - Combinations of Allelopathic Crop Extracts Reduce Digitaria spp. and Setaria faberi Seed Germination
\end{cvitems}}
\end{cventries}

\hypertarget{technical-skills}{%
\section{Technical Skills}\label{technical-skills}}

\begin{table}[!h]
\centering\begingroup\fontsize{10}{12}\selectfont

\begin{tabular}{>{\centering\arraybackslash}p{9 cm}>{\centering\arraybackslash}p{9 cm}>{}p{9 cm}}
\toprule
\textcolor[HTML]{000000}{\textbf{Dry Lab}} & \textcolor[HTML]{000000}{\textbf{Wet Lab}}\\
\midrule
\textcolor[HTML]{414141}{Snapgene Viewer, BLAST, Ensembl, Microsoft Office, R (ggplot2, agricolae, RMarkdown, corrplot, tidyverse, and other packages)} & \textcolor[HTML]{414141}{Tissue Culture, Protoplasting, Nucleic acid isolation, Gel electrophoresis, PCR, quantitative PCR, RT-PCR, Experimental design}\\
\bottomrule
\end{tabular}
\endgroup{}
\end{table}

\hypertarget{experience}{%
\section{Experience}\label{experience}}

\begin{cventries}
    \cventry{Mydecine Innovations Group Inc.}{Research Associate}{Denver, Colorado}{October 2020 → April 2022}{\begin{cvitems}
\item Conceived the design of intellectual property related to increasing the rate biosynthesis of secondary metabolites in a filamentous fungus to be used as a heterologous host. Listed as a co-inventor on the provisional patent.
\item Designed and implemented experiments related to nucleic acid isolation, molecular identification of species, gene expression analysis, stimulation of defense compound biosynthesis, cloning of metabolite biosynthesis cassettes, protoplasting, and CRISPR-Cas9 genome editing.
\item Created mutants of a filamentous fungus to serve as platforms for ectopic metabolite production
\item Facilitated the acquisition of scientific equipment and overall operationality of a 7500 ft\^2 laboratory
\item Trained colleagues and one intern in fungal tissue culture, molecular biology techniques, and basic use of R for data visualization and statistical analysis
\item Created R Markdown guides on how to implement data visualization and analysis in R
\end{cvitems}}
    \cventry{University of Connecticut }{Graduate Research Assistant}{Storrs, Connecticut}{August 2018 → August 2020}{\begin{cvitems}
\item Successfully defended thesis on genetic regulation of cannabinoid biosynthesis in Cannabis sativa.
\item Designed and executed experiments in university greenhouses and commercial grow facilities.
\item Implemented quantitative PCR to measure gene expression levels of genes in the cannabinoid biosynthesis pathway and quantified cannabinoid content using HPLC
\item Analyzed and visualized data in R to facilitate comprehension in invited talks and manuscripts.
\item Obtained controlled substances research licenses for laboratory, wrote standard operating procedures for the sampling, transport, and secure storage of controlled substances.
\end{cvitems}}
    \cventry{University of Connecticut }{Graduate Mentor}{Storrs, Connecticut}{August 2018 → August 2020}{\begin{cvitems}
\item Trained undergraduate and graduate students in molecular biology techniques.
\item Mentored one student to win a grant, design and execute an experiment, and present findings at a university research forum.
\end{cvitems}}
    \cventry{University of Connecticut }{Teaching Assistant}{Storrs, Connecticut}{August 2018 → August 2020}{\begin{cvitems}
\item Lectured one laboratory section of woody plant identification. Led 15 students through university campus to teach students characteristics of plants. Graded assignments and recorded grades.
\end{cvitems}}
    \cventry{University of Connecticut }{Undergraduate Researcher}{Storrs, Connecticut}{August 2016 → May 2018}{\begin{cvitems}
\item Awarded \$8000 in grant funding to phenotypically survey North America’s largest Aronia germplasm collection. 
\item Evaluated sugar and acidity content of fruit juice from 120 Aronia genotypes, commonly referred to as chokeberry. Identified specific genotypes with desirable Brix°: titratable acidity ratios to be used as parentage for breeding varieties with more palatable fruits. 
\item Presented findings in posters at national and international conferences.
\end{cvitems}}
\end{cventries}

\hypertarget{publications}{%
\section{Publications}\label{publications}}

\leavevmode\vadjust pre{\hypertarget{ref-apicella_delineating_2022}{}}%
\textbf{Apicella, P.}, Ma, G., Ma, Y., \& Berkowitz, G. A.
The Cannabis Jasmonate-Independent Homeodomain Zipper Family IV Gene 
HDG5 Functions in Trichome Morphogenesis and Involves 
in Immune response in transgenic \textsc{Nicotiana tabacum L.} 
\emph{manuscript in preparation}

\leavevmode\vadjust pre{\hypertarget{ref-apicella_delineating_2022}{}}%
\textbf{Apicella, P.}, Ma, G., Ma, Y., \& Berkowitz, G. A. (2022).
Delineating genetic regulation of cannabinoid biosynthesis during female
flower development in \textsc{Cannabis sativa }. \emph{Plant Direct},
\emph{6}(6). \url{https://doi.org/10.1002/pld3.412}

\leavevmode\vadjust pre{\hypertarget{ref-haiden_overexpression_2022}{}}%
Haiden, S. R., \textbf{Apicella, P.}, Ma, Y., \& Berkowitz, G. A. (2022).
Overexpression of CsMIXTA, a Transcription Factor from Cannabis sativa,
Increases Glandular Trichome Density in Tobacco Leaves. \emph{Plants},
\emph{11}(11), 1519. \url{https://doi.org/10.3390/plants11111519}

\leavevmode\vadjust pre{\hypertarget{ref-ma_genome-wide_2022}{}}%
Ma, G., Zelman, A. K., \textbf{Apicella, P.}, \& Berkowitz, G. (2022).
Genome-Wide Identification and Expression Analysis of Homeodomain
Leucine Zipper Subfamily IV (HD-ZIP IV) Gene Family in Cannabis sativa
L. \emph{Plants}, \emph{11}(10), 1307.
\url{https://doi.org/10.3390/plants11101307}

\leavevmode\vadjust pre{\hypertarget{ref-mcgehee_first_2019}{}}%
McGehee, C. S., \textbf{Apicella, P.}, Raudales, R., Berkowitz, G., Ma, Y.,
Durocher, S., \& Lubell, J. (2019). First Report of Root Rot and Wilt
Caused by Pythium myriotylum on Hemp ( Cannabis sativa ) in the United
States. \emph{Plant Disease}, \emph{103}(12), 3288--3288.
\url{https://doi.org/10.1094/PDIS-11-18-2028-PDN}

\leavevmode\vadjust pre{\hypertarget{ref-mahoney_adventitious_2018}{}}%
Mahoney, J. D., \textbf{Apicella, P.}, \& Brand, M. H. (2018). Adventitious
shoot regeneration from in vitro leaves of Aronia mitschurinii and
cotyledons of closely related Pyrinae taxa. \emph{Scientia
Horticulturae}, \emph{237}, 135--141.
\url{https://doi.org/10.1016/j.scienta.2018.03.062}

\hypertarget{invited-talks}{%
\section{Invited Talks}\label{invited-talks}}

\documentclass{article}
\usepackage[sfdefault]{roboto}

\leavevmode\vadjust pre{\hypertarget{ref-mahoney_adventitious_2018}{}}%
\textbf{Apicella P.} "The Plant Science of \emph{Cannabis sativa}", Tower Hill 
Botanical Garden, Virtal oral presentation. March 2022.

\leavevmode\vadjust pre{\hypertarget{ref-mahoney_adventitious_2018}{}}%
\textbf{Apicella P.} "The Plant Science of \emph{Cannabis sativa}", 
Groton Garden Club, Virtal oral presentation. May 2021.

\leavevmode\vadjust pre{\hypertarget{ref-mahoney_adventitious_2018}{}}%
\textbf{Apicella P.} "The Plant Science of \emph{Cannabis sativa}", Tower Hill 
Botanical Garden, Oral presentation. March 2020.

\leavevmode\vadjust pre{\hypertarget{ref-mahoney_adventitious_2018}{}}%
\textbf{Apicella P.}, Ma Y., Schultz T., Ferrarese R., Picard R., 
Barolli S., and Berkowitz G. "Looking Under The Hood Of The Cannabis Plant: 
A Molecular Evaluation Of Cannabinoid Production", Emerald Conference,
Oral presentation. February 2020.

\leavevmode\vadjust pre{\hypertarget{ref-mahoney_adventitious_2018}{}}%
\textbf{Apicella P.} and Berkowitz G. "The Plant Science of \emph{Cannabis sativa}", 
Middlesex Community College, Oral presentation. New York, New York. October 2019.

\leavevmode\vadjust pre{\hypertarget{ref-mahoney_adventitious_2018}{}}%
\textbf{Apicella P.} and Berkowitz G. "The Plant Science of \emph{Cannabis sativa}", Metropolitan 
Horticultural Society, Oral presentation. New York, New York. September 2019.

\leavevmode\vadjust pre{\hypertarget{ref-mahoney_adventitious_2018}{}}%
\textbf{Apicella P.} and Berkowitz G. "Association of a prenyltransferase (GOT) with THCA 
production in medical marijuana", American Society of Plant Biology, 
Brief oral presentation. San Jose, California. August 2019.

\leavevmode\vadjust pre{\hypertarget{ref-mahoney_adventitious_2018}{}}%
\textbf{Apicella P.} and Berkowitz G. "The Plant Science of \emph{Cannabis sativa}", 
Northeast Food And Drug Officials Association Meeting, Oral presentation. 
Northampton, Massachusetts. May 2019.

\pagebreak



\end{document}
